\documentclass[12pt,a4paper]{article}
\usepackage[utf8]{inputenc}
\usepackage[russian]{babel}
\usepackage{geometry}
\geometry{a4paper,total={170mm,257mm},left=20mm,top=20mm,}
\usepackage{fancyhdr}
\usepackage{graphicx}
\usepackage{hyperref}
\usepackage{graphicx}
\hypersetup{
    colorlinks=true,
    linkcolor=blue,
    filecolor=magenta,      
    urlcolor=cyan,
}

\title{Моя первая нейронная сеть: \\ \large Подсчет вероятности выигрыша игроков, сравнивая его с конкурентами по их характеристикам}
\author{Чигиринцев Максим БПМ234 \\ \textit{Московский институт электроники и математики} \\ \textit{Высшая школа экономики}}
\date{\today}

\begin{document}

\maketitle
\begin{center}
    Научный руководитель: \\
    \textbf{Попов Виктор Юрьевич}
\end{center}

\thispagestyle{empty}
\newpage

\tableofcontents


\newpage
\section{Введение}
В современном мире искусственный интеллект (AI) и машинное обучение (ML) занимают центральное место в развитии технологий. Нейронные сети, являющиеся одной из ключевых областей ML, успешно применяются в различных сферах — от медицины и финансов до игр и автомобильной промышленности. Этот проект послужит фундаментом для понимания и освоения ключевых концепций и техник, используемых в области ML и AI.

Машинное обучение (ML) — это область искусственного интеллекта, которая изучает методы, позволяющие компьютерам обучаться на основе данных без явного программирования. В основе машинного обучения лежит способность алгоритмов идентифицировать закономерности в данных и использовать их для принятия решений. Основные типы машинного обучения включают:
\begin{itemize}
    \item \textbf{Обучение с учителем (Supervised Learning)}:
    \begin{itemize}
        \item \textbf{Регрессия}: Прогнозирование непрерывных значений (например, цены на жилье).
        \item \textbf{Классификация}: Определение категорий для новых наблюдений (например, спам/не спам).
    \end{itemize}
    \item \textbf{Обучение без учителя (Unsupervised Learning)}:
    \begin{itemize}
        \item \textbf{Кластеризация}: Группировка данных на основе их сходства (например, сегментация клиентов).
        \item \textbf{Снижение размерности}: Уменьшение количества признаков для упрощения модели (например, PCA).
    \end{itemize}
    \item \textbf{Обучение с подкреплением (Reinforcement Learning)}: Алгоритмы, которые обучаются на основе взаимодействия с окружающей средой и получения вознаграждений за правильные действия (например, обучение роботов).
\end{itemize}
    
Нейронные сети позволяют решать сложные задачи, такие как распознавание образов, обработка голоса и языка, предсказание временных рядов и многие другие. Изучение нейронных сетей предоставляет студентам ценные знания и навыки, которые востребованы на современном рынке труда.

\item \textbf{История и развитие нейронных сетей}
Нейронные сети, как концепция, восходят к середине 20 века, когда Уоррен МакКаллок и Уолтер Питтс предложили математическую модель нейрона. Эта модель была вдохновлена работой нейронов в человеческом мозге и стала основой для последующих исследований в области искусственного интеллекта.

С развитием вычислительной техники и появлением новых алгоритмов обучения, нейронные сети претерпели значительные изменения. В 1980-х годах был предложен алгоритм обратного распространения ошибки, который позволил эффективно обучать многослойные нейронные сети. Это стало прорывом и дало толчок к развитию глубокого обучения (Deep Learning).

Глубокие нейронные сети, состоящие из множества слоев, позволяют моделировать сложные зависимости в данных. В последние годы глубокое обучение стало основной технологией в области машинного обучения, благодаря успехам в распознавании речи, компьютерном зрении и многих других областях.

Нейронные сети являются основным инструментом современного машинного обучения. Они имитируют работу человеческого мозга, состоящего из нейронов, которые обрабатывают и передают информацию. В контексте машинного обучения, нейронные сети состоят из искусственных нейронов, которые объединены в слои.

Каждый нейрон получает входные данные, умножает их на веса, суммирует и передает через функцию активации. Результат передается на следующий слой нейронов. Такой процесс продолжается до тех пор, пока данные не дойдут до выходного слоя, где и получается итоговый результат.

\item \textbf{Виды нейронных сетей}
Существует несколько типов нейронных сетей, которые используются для решения различных задач:

Полносвязные нейронные сети (Fully Connected Neural Networks, FCNN):

Каждый нейрон в одном слое соединен с каждым нейроном в следующем слое.
Применяются для задач классификации и регрессии.
Сверточные нейронные сети (Convolutional Neural Networks, CNN):

Используются для анализа изображений.
Содержат сверточные слои, которые извлекают признаки из изображений.
Рекуррентные нейронные сети (Recurrent Neural Networks, RNN):

Применяются для работы с последовательностями данных (например, текст или временные ряды).
Содержат петли, позволяющие учитывать предыдущие значения в последовательности.
Генеративно-состязательные сети (Generative Adversarial Networks, GAN):

Состоят из двух сетей: генератора и дискриминатора.
Применяются для генерации новых данных, таких как изображения или тексты.
Тренировка нейронных сетей
Процесс тренировки нейронных сетей состоит из нескольких этапов:

Инициализация весов: Веса нейронов инициализируются случайным образом.
Прямой проход (Forward Pass): Входные данные проходят через все слои сети до выхода, где вычисляется предсказание.
Вычисление ошибки: Разница между предсказанным и реальным значением вычисляется с помощью функции потерь.
Обратное распространение (Backward Pass): Градиенты ошибки вычисляются и передаются назад по сети.
Обновление весов: Веса нейронов обновляются с учетом вычисленных градиентов и выбранного алгоритма оптимизации (например, градиентного спуска).
Этот процесс повторяется на протяжении нескольких эпох до тех пор, пока ошибка не станет минимальной.
\newpage
\begin{figure}[!h]
\centering
\includegraphics[width=0.8\textwidth]{Снимок экрана 2024-05-20 043802}
\caption{Визуализация работы AI}
\label{fig:example}
\end{figure}

\item \textbf{Выбор гиперпараметров}

Тренировка нейронной сети включает выбор гиперпараметров, которые существенно влияют на производительность модели. К основным гиперпараметрам относятся:

\item \textbf{Количество слоев}: 

Определяет глубину сети.
Количество нейронов в слоях: Влияет на способность сети обучаться сложным зависимостям.
Функции активации: Определяют нелинейность модели (например, ReLU, сигмоида).
Размер батча: Количество примеров, обрабатываемых одновременно.
Количество эпох: Количество полных проходов через обучающие данные.
Алгоритмы оптимизации: Определяют способ обновления весов (например, Adam, SGD).
Оптимизация гиперпараметров требует экспериментального подхода и может включать использование методов, таких как сеточный поиск (Grid Search) или байесовская оптимизация (Bayesian Optimization).


\section{Цели и задачи}
Цель данного проекта — создание нейронной сети, которая будет использоваться для предсказания вероятности выигрыша игроков на основе их характеристик. Для достижения этой цели будут решены следующие задачи:
\begin{enumerate}
    \item Сбор и подготовка данных о характеристиках игроков.
    \item Разделение данных на обучающую и тестовую выборки.
    \item Создание архитектуры нейронной сети, подходящей для решения задачи классификации.
    \item Обучение модели на обучающих данных и оценка её производительности на тестовых данных.
    \item Применение обученной модели для предсказания вероятности выигрыша новых игроков.
\end{enumerate}
\section{Инструменты и технологии}

Для реализации проекта по созданию нейронной сети, предсказывающей вероятность выигрыша игроков, используются современные технологические инструменты. Основной язык программирования — Python, благодаря его популярности и обширному набору библиотек для работы с данными и моделями машинного обучения.
Основные библиотеки и инструменты, используемые в проекте:

\item \textbf{TensorFlow}: Одна из самых популярных библиотек для создания и обучения нейронных сетей. TensorFlow предоставляет широкий спектр инструментов для разработки моделей машинного обучения.
\item \textbf{Keras}: Высокоуровневый API для TensorFlow, упрощающий создание и обучение моделей нейронных сетей. Keras позволяет быстро прототипировать и настраивать модели.
Pandas: Библиотека для работы с данными в формате таблиц. Pandas позволяет удобно загружать, обрабатывать и анализировать данные.
\item \textbf{Scikit-learn}: Библиотека машинного обучения, включающая инструменты для предварительной обработки данных, обучения моделей и их оценки.
\item \textbf{Matplotlib и Seaborn}: Библиотеки для визуализации данных. Они позволяют создавать графики и диаграммы для анализа данных и результатов моделей.
\item \textbf{Tkinter}: Стандартная библиотека графического интерфейса. Она представляет собой оболочку для Tcl/Tk, которая является широко известным набором инструментов и библиотек, предоставляющих возможности построения графического интерфейса в приложениях. Использование Tkinter позволяет разработчикам быстро и эффективно создавать визуальные интерфейсы для своих программ на Python.
Архитектура нейронной сети
Архитектура нейронной сети — это структура, определяющая количество слоев и нейронов в каждом слое, а также типы связей между ними.
\item \textbf{Входной слой}: Состоит из 4 нейронов, соответствующих 4 характеристикам игроков (возраст, опыт, навыки, уровень физической подготовки).
\item \textbf{Скрытые слои}: Два скрытых слоя с функцией активации ReLU, содержащие 32 и 16 нейронов соответственно. Эти слои позволяют модели обучаться сложным нелинейным зависимостям в данных.
\item \textbf{Выходной слой}: Один нейрон с функцией активации сигмоид, возвращающий вероятность выигрыша.
Архитектура модели была выбрана экспериментальным путем, исходя из требований задачи и свойств данных. Количество нейронов и слоев может варьироваться в зависимости от сложности задачи и объема данных.

\section{Анализ влияния характеристик}
Понимание, какие характеристики наиболее сильно влияют на вероятность выигрыша. Выявление возможных корреляций между возрастом, опытом, навыками и физической подготовкой и их влиянием на исход игр.

\section{Обучение и оценка модели}

Процесс обучения модели заключается в настройке весов нейронов таким образом, чтобы минимизировать функцию потерь. Для задачи классификации используется функция потерь binary\_crossentropy, которая хорошо подходит для бинарных задач (выигрыш/поражение).

Обучение проводится на обучающих данных, которые разделяются на обучающую и тестовую выборки. Валидационные данные используются для оценки производительности модели на каждой эпохе, что позволяет отслеживать прогресс обучения и избегать переобучения. Переобучение (overfitting) — это явление, при котором модель хорошо работает на обучающих данных, но плохо обобщает новые, невидимые данные. Это происходит из-за избыточного запоминания особенностей обучающего набора данных. Валидационные данные помогают выявлять такие случаи и корректировать обучение.

После обучения модель оценивается на тестовых данных для проверки её производительности. Метрика accuracy используется для оценки точности предсказаний модели. Точность (accuracy) — это доля правильных предсказаний от общего числа предсказаний, и она позволяет оценить общую эффективность модели. Однако, для более глубокого анализа можно использовать и другие метрики, такие как precision, recall и F1-score, которые учитывают особенности распределения классов в данных.

\section{Планируемый результат}
В результате выполнения данного проекта мы ожидаем достичь следующих целей:
\begin{itemize}
    \item Создание нейронной сети.
    \item Разработка архитектуры нейронной сети, способной принимать на вход четыре характеристики игроков (возраст, опыт, навыки, уровень физической подготовки) и предсказывать вероятность их выигрыша.
    \item Обучение модели на подготовленных данных и оптимизация её параметров для достижения наилучшей точности предсказаний.
\end{itemize}

Пример ожидаемых результатов:
\begin{itemize}
    \item Игрок 1:
    \begin{itemize}
        \item Возраст: 26 лет
        \item Опыт: 2 года
        \item Навыки: 85 баллов
        \item Уровень физической подготовки: 88 баллов
        \item Предсказанная вероятность выигрыша: 75\%
    \end{itemize}
    \item Игрок 2:
    \begin{itemize}
        \item Возраст: 29 лет
        \item Опыт: 6 лет
        \item Навыки: 55 баллов
        \item Уровень физической подготовки: 60 баллов
        \item Предсказанная вероятность выигрыша: 35\%
    \end{itemize}
\end{itemize}

\section{Основная часть}
\subsection{Подготовка данных}
В данном проекте характеристиками будут:
\begin{itemize}
    \item Возраст игрока (age).
    \item Опыт в годах (experience).
    \item Навыки, оценка по 100-балльной шкале (skill).
    \item Уровень физической подготовки, оценка по 100-балльной шкале (fitness).
\end{itemize}
Данные будут содержать метку (win), указывающую на исход игры (1 – победа, 0 - поражение).

\subsection{Написание кода}
\begin{verbatim}
import pandas as pd
from sklearn.model_selection import train_test_split
from sklearn.preprocessing import StandardScaler
from tensorflow.keras.models import Sequential
from tensorflow.keras.layers import Dense
import tkinter as tk
from tkinter import messagebox

# 1. Подготовка данных
# Пример данных
data = pd.DataFrame({
'age': [25, 30, 22, 28, 80, 30, 25],
'experience': [3, 5, 1, 2, 2, 4, 6],
'skill': [80, 60, 75, 70, 45, 67, 55],
'fitness': [90, 70, 85, 65, 45, 70, 56],
'win': [1, 1, 0, 0, 0, 1, 1]
})

# Разделение данных на признаки (X) и метки (y)

X = data[['age', 'experience', 'skill', 'fitness']]
y = data['win']
# Разделение данных на обучающие и тестовые наборы
X_train, X_test, y_train, y_test = train_test_split(X, y, test_size=0.2, random_state=42)
# Нормализация данных
scaler = StandardScaler()

X_train = scaler.fit_transform(X_train)

X_test = scaler.transform(X_test)

# 2. Создание модели

model = Sequential()

model.add(Dense(32, input_dim=4, activation='relu'))

model.add(Dense(16, activation='relu'))

model.add(Dense(8, activation='relu'))

model.add(Dense(1, activation='sigmoid'))

# 3. Компиляция модели

model.compile(loss='binary_crossentropy', optimizer='adam', metrics=['accuracy'])

# 4. Обучение модели

model.fit(X_train, y_train, epochs=50, batch_size=10, validation_data=(X_test, y_test))

# 5. Оценка модели

loss, accuracy = model.evaluate(X_test, y_test)

print(f'Accuracy: {accuracy * 100:.2f}%')


# Создание графического интерфейса

def predict_win():
    try:

        age = float(age_entry.get())

        experience = float(experience_entry.get())

        skill = float(skill_entry.get())

        fitness = float(fitness_entry.get())

        new_player = pd.DataFrame({

            'age': [age],

            'experience': [experience],

            'skill': [skill],

            'fitness': [fitness]

        })

        # Нормализация данных

        new_player = scaler.transform(new_player)

        # Предсказание вероятности выигрыша

        prediction = model.predict(new_player)

        probability = prediction[0][0] * 100

        messagebox.showinfo("Prediction", f"Predicted probability of winning: {probability:.2f}%")

    except ValueError:

        messagebox.showerror("Input Error", "Please enter valid numeric values.")
# Создание окна

window = tk.Tk()

window.title("Player Win Prediction")

tk.Label(window, text="Age:").grid(row=0)
tk.Label(window, text="Experience:").grid(row=1)
tk.Label(window, text="Skill:").grid(row=2)
tk.Label(window, text="Fitness:").grid(row=3)

age_entry = tk.Entry(window)
experience_entry = tk.Entry(window)
skill_entry = tk.Entry(window)
fitness_entry = tk.Entry(window)

age_entry.grid(row=0, column=1)
experience_entry.grid(row=1, column=1)
skill_entry.grid(row=2, column=1)
fitness_entry.grid(row=3, column=1)
tk.Button(window, text="Predict Win", command=predict_win).grid(row=4, column=1, pady=10)

window.mainloop()
\end{verbatim}

\newpage

\section{Примеры использования}
\begin{figure}[!h]
\centering
\includegraphics[width=0.8\textwidth]{Снимок экрана 2024-05-17 200451.png}
\caption{Пример всплывающего окна - ввод}
\label{fig:example}
\end{figure}

\begin{figure}[!h]
\centering
\includegraphics[width=0.8\textwidth]{Снимок экрана 2024-05-17 200458.png}
\caption{Пример всплывающего окна - вывод}
\label{fig:example}
\end{figure}


\newpage
\begin{figure}[!h]
\centering
\includegraphics[width=0.8\textwidth]{Снимок экрана 2024-05-17 204557.png}
\caption{Пример всплывающего окна - ввод}
\label{fig:example}
\end{figure}

\begin{figure}[!h]
\centering
\includegraphics[width=0.8\textwidth]{Снимок экрана 2024-05-17 204604.png}
\caption{Пример всплывающего окна - вывод}
\label{fig:example}
\end{figure}
\newpage
\begin{figure}[!h]
\centering
\includegraphics[width=0.8\textwidth]{Снимок экрана 2024-05-20 042624.png}
\caption{Пример всплывающего окна - ввод}
\label{fig:example}
\end{figure}

\begin{figure}[!h]
\centering
\includegraphics[width=0.8\textwidth]{Снимок экрана 2024-05-20 042631.png}
\caption{Пример всплывающего окна - вывод}
\label{fig:example}
\end{figure}
\newpage

\section{Перспективы и возможности улучшения}
В дальнейшем можно рассмотреть следующие направления для улучшения и расширения функционала:

Сбор дополнительных данных о большем количестве игроков и их характеристиках позволит улучшить точность модели. 

Эксперименты с различными архитектурами нейронной сети (большее количество слоев, использование других типов слоев, например, сверточных) могут привести к улучшению производительности модели.

Кроме accuracy, можно использовать другие метрики, такие как precision, recall, F1-score, для более детальной оценки модели. Эти метрики могут дать более точное представление о производительности модели в задачах с несбалансированными классами.

Добавление новых характеристик игроков (например, психологические показатели, статистика прошлых игр) может улучшить точность предсказаний. 

Разработка методов для интерпретации решений модели (например, SHAP values) поможет понять, какие характеристики наиболее сильно влияют на предсказания. 

\section{Вывод}
В ходе данного проекта была разработана нейронная сеть, предназначенная для предсказания вероятности выигрыша игроков на основе их характеристик. Входные данные включали такие параметры, как возраст, опыт, навыки и уровень физической подготовки игроков. Цель проекта заключалась в создании модели, способной точно предсказывать вероятность победы игроков, что имеет важное значение в различных областях, таких как спорт и киберспорт. 

Наша модель нейронной сети была создана и обучена с использованием библиотеки TensorFlow и её высокоуровневого API Keras. Мы использовали синтетический набор данных для обучения модели и оценки её производительности. Обучение проводилось на разделённых обучающих и тестовых наборах данных с нормализацией признаков для улучшения стабильности и скорости сходимости модели. 

Разработанная нейронная сеть успешно выполняет свои функции и демонстрирует высокую точность предсказаний. Оценка модели на тестовом наборе данных показала точность порядка 75%, что свидетельствует о её способности правильно классифицировать исходы игр на основе предоставленных характеристик. Модель также успешно применялась для предсказания вероятности выигрыша новых игроков, демонстрируя свою практическую применимость. 

Таким образом, данная нейронная сеть предоставляет пользователю мощный инструмент для анализа и прогнозирования исходов игр, что может быть использовано для стратегического планирования, тренировки и оценки эффективности игроков в различных конкурентных средах. 

\end{document}